%%%%%%%%%%%%%%%%%%%%%%%%%%%%%%%%%%%%%%%%%%%%%%%%%%%%%%%%%%%%
%%  This Beamer template was created by Cameron Bracken.
%%  Anyone can freely use or modify it for any purpose
%%  without attribution.
%%
%%  Last Modified: January 9, 2009
%%

\documentclass[xcolor=x11names,compress]{beamer}

%% General document %%%%%%%%%%%%%%%%%%%%%%%%%%%%%%%%%%
\usepackage{graphicx}
\usepackage{tikz}
\usepackage[canadian]{babel}
\usepackage[utf8]{inputenc}
\usepackage{amsmath,amssymb}
\usepackage{mathtools}
\usepackage[ruled,vlined,linesnumbered]{algorithm2e}
\usepackage{epstopdf}
\usepackage{hyperref}
\usepackage[export]{adjustbox}
\usepackage{animate}
\graphicspath{{img/}}
%%%%%%%%%%%%%%%%%%%%%%%%%%%%%%%%%%%%%%%%%%%%%%%%%%%%%%


%% Beamer Layout %%%%%%%%%%%%%%%%%%%%%%%%%%%%%%%%%%
\useoutertheme[subsection=false,shadow]{miniframes}
\useinnertheme{default}
\usefonttheme{serif}
\usepackage{palatino}
\urlstyle{same}
\setbeamerfont{title like}{shape=\scshape}
\setbeamerfont{frametitle}{shape=\scshape}

\setbeamercolor*{lower separation line head}{bg=DeepSkyBlue4} 
\setbeamercolor*{normal text}{fg=black,bg=white} 
\setbeamercolor*{alerted text}{fg=red} 
\setbeamercolor*{example text}{fg=black} 
\setbeamercolor*{structure}{fg=black} 
 
\setbeamercolor*{palette tertiary}{fg=black,bg=black!10} 
\setbeamercolor*{palette quaternary}{fg=black,bg=black!10} 

\renewcommand{\(}{\begin{columns}}
\renewcommand{\)}{\end{columns}}
\newcommand{\<}[1]{\begin{column}{#1}}
\renewcommand{\>}{\end{column}}

\addtobeamertemplate{navigation symbols}{}{%
	\usebeamerfont{footline}%
	\usebeamercolor[fg]{footline}%
	\hspace{1.5ex}%
	\insertframenumber/\inserttotalframenumber
}
%%%%%%%%%%%%%%%%%%%%%%%%%%%%%%%%%%%%%%%%%%%%%%%%%%


\begin{document}


%%%%%%%%%%%%%%%%%%%%%%%%%%%%%%%%%%%%%%%%%%%%%%%%%%%%%%
%%%%%%%%%%%%%%%%%%%%%%%%%%%%%%%%%%%%%%%%%%%%%%%%%%%%%%
\section{\scshape Introduction}
\begin{frame}
\title{Introduction to Machine Learning}
\subtitle{Linear Regression}
\author{
	Davi Frossard\\
	{\it Federal University of Espirito Santo \\ University of Toronto}\\
}
\date{
    \vspace{-2em}\\
    \includegraphics[width=0.5\columnwidth]{linreg.png}\\[-1ex]
    {\tiny Credit: {\itshape \url{http://bit.ly/2cwKaex}}}
    \\
	\today
}
\titlepage
\end{frame}

%%%%%%%%%%%%%%%%%%%%%%%%%%%%%%%%%%%%%%%%%%%%%%%%%%%%%%
%%%%%%%%%%%%%%%%%%%%%%%%%%%%%%%%%%%%%%%%%%%%%%%%%%%%%%
\begin{frame}{Summary}
\tableofcontents
\end{frame}

%%%%%%%%%%%%%%%%%%%%%%%%%%%%%%%%%%%%%%%%%%%%%%%%%%%%%%
%%%%%%%%%%%%%%%%%%%%%%%%%%%%%%%%%%%%%%%%%%%%%%%%%%%%%%
\subsection{Overview}
\begin{frame}{Linear Regression}
	\begin{itemize}
		\item Find a relationship between a dependent variable and a set of independent variables.
		\item In its simplest form, consists of finding a function $\boldsymbol{y} = w.\boldsymbol{x} + b$ that fits the data.
		\begin{center}
			\adjincludegraphics[width=0.7\columnwidth]{lreg_ex}
		\end{center}
	\end{itemize}
\end{frame}

%%%%%%%%%%%%%%%%%%%%%%%%%%%%%%%%%%%%%%%%%%%%%%%%%%%%%%
%%%%%%%%%%%%%%%%%%%%%%%%%%%%%%%%%%%%%%%%%%%%%%%%%%%%%%
\subsection{Real Estate Example}
\begin{frame}{Real Estate Example}
	\begin{itemize}
		\item We may want to predict the cost of a house given its area.
		\item Gather real estate data from many listings and fit a curve.
	\end{itemize}
	\begin{center}
	   \includegraphics[width=0.6\columnwidth]{real_estate}\\[-1ex]
	   {\tiny Credit: {\itshape \url{http://efs.com.sg/common-pitfalls-of-real-estate-investing-and-how-to-avoid-them/}}}
	\end{center}
\end{frame}

%%%%%%%%%%%%%%%%%%%%%%%%%%%%%%%%%%%%%%%%%%%%%%%%%%%%%%
%%%%%%%%%%%%%%%%%%%%%%%%%%%%%%%%%%%%%%%%%%%%%%%%%%%%%%
\subsection{The Problem}
\begin{frame}{The Problem}
	\begin{itemize}
		\item We want the parameters $w$ and $b$ in $\boldsymbol{y} = w.\boldsymbol{x} + b$ such that the \textit{distance} between the points and the fitted line is minimized.
		\item In other words, we want to minimize the average error.
		\begin{center}
			\adjincludegraphics[width=0.6\columnwidth]{lreg}
		\end{center}
	\end{itemize}
\end{frame}

%%%%%%%%%%%%%%%%%%%%%%%%%%%%%%%%%%%%%%%%%%%%%%%%%%%%%%
%%%%%%%%%%%%%%%%%%%%%%%%%%%%%%%%%%%%%%%%%%%%%%%%%%%%%%
\begin{frame}{The Problem}
	\begin{itemize}
		\item You might have seen it before as a solution to a linear problem, obtained by setting the derivative of the error to zero:
	\end{itemize}
	\begin{gather*}
		\beta = (X^TX)^{-1}X^TY
	\end{gather*}
	\begin{itemize}
		\item Inversion does not scale well with the size of the matrix ($\thickapprox O(n^3)$).
		\item In our realm of big data, this rapidly gets out of hand.
		\item We need \textbf{Gradient Descent}.
	\end{itemize}
\end{frame}


%%%%%%%%%%%%%%%%%%%%%%%%%%%%%%%%%%%%%%%%%%%%%%%%%%%%%%
%%%%%%%%%%%%%%%%%%%%%%%%%%%%%%%%%%%%%%%%%%%%%%%%%%%%%%
\section{\scshape Gradient Descent}
\begin{frame}{Gradient Descent}
	\begin{itemize}
		\item Used to minimize \textbf{differentiable} functions like Newton's method, but does not require second order derivatives (\textit{Hessian}).
		\item Recall that at any point in a surface, the gradient at that point always points in the direction of \textbf{steepest ascent} and at a minimum, it will be \textbf{zero}.
		\item Like a ball rolling to the bottom of a bowl.
		\begin{center}
			\includegraphics[width=0.9\columnwidth]{sgd_optimal.png}\\[-1ex]
			{\tiny Credit: {\itshape \url{http://iamtrask.github.io/2015/07/27/python-network-part2/}}}
		\end{center}
	\end{itemize}
\end{frame}


%%%%%%%%%%%%%%%%%%%%%%%%%%%%%%%%%%%%%%%%%%%%%%%%%%%%%%
%%%%%%%%%%%%%%%%%%%%%%%%%%%%%%%%%%%%%%%%%%%%%%%%%%%%%%
\begin{frame}{Gradient Descent}
	\begin{itemize}
		\item Take \textbf{weighted} steps at each time in the direction of \textbf{steepest descent}.
		\item We call this weight the \textbf{learning rate} ($\alpha$). I.e.:
	\end{itemize}
	\begin{gather*}
		x \leftarrow x - \alpha \frac{F(x)}{x}
	\end{gather*}
	\begin{center}
		\animategraphics[controls,width=0.47\linewidth]{12}{sgd_anim/sgd-}{1}{51}\\
		{\tiny Credit: {\itshape \url{http://scs.ryerson.ca/~aharley/neural-networks/}}}
	\end{center}
\end{frame}

%%%%%%%%%%%%%%%%%%%%%%%%%%%%%%%%%%%%%%%%%%%%%%%%%%%%%%
%%%%%%%%%%%%%%%%%%%%%%%%%%%%%%%%%%%%%%%%%%%%%%%%%%%%%%
\subsection{$\alpha$ Sensitivity}
\begin{frame}{$\alpha$ Sensitivity}
	\begin{itemize}
		\item $\alpha$ is a sensitive parameter in the sense that if set too high, may cause divergence.
		\item Many variants of gradient descent propose solutions to this problem.
	\end{itemize}
	\begin{center}
		\animategraphics[controls,width=0.47\linewidth]{12}{sgd_anim_div/sgd-}{1}{51}\\
		{\tiny Credit: {\itshape \url{http://scs.ryerson.ca/~aharley/neural-networks/}}}
	\end{center}
\end{frame}

%%%%%%%%%%%%%%%%%%%%%%%%%%%%%%%%%%%%%%%%%%%%%%%%%%%%%%
%%%%%%%%%%%%%%%%%%%%%%%%%%%%%%%%%%%%%%%%%%%%%%%%%%%%%%
\subsection{Gradient Descent in Linear Regression}
\begin{frame}{Gradient Descent in Linear Regression}
	\begin{itemize}
		\item To learn our linear regression model we need to set the values of $w$ and $b$ in $\boldsymbol{y} = w.\boldsymbol{x} + b$ that minimize the average error.
		\item If we use L2-Distance, our error becomes: 
	\end{itemize}
	\begin{gather*}
	E(x,w,b) = \dfrac{1}{n}\sum_{i = 0}^n \Big(y_i - (wx_i + b)\Big)^2
	\end{gather*}
	\begin{itemize}
		\item In \textbf{batch} gradient descent we evaluate the gradient over all $\boldsymbol{x}$ in order to decide our next step. I.e.:
	\end{itemize}
	\begin{gather*}
	w \leftarrow w - \alpha \frac{\partial E(x,w,b)}{\partial w}\\
	b \leftarrow b - \alpha \frac{\partial E(x,w,b)}{\partial b}
	\end{gather*}
\end{frame}

%%%%%%%%%%%%%%%%%%%%%%%%%%%%%%%%%%%%%%%%%%%%%%%%%%%%%%
%%%%%%%%%%%%%%%%%%%%%%%%%%%%%%%%%%%%%%%%%%%%%%%%%%%%%%
\subsection{Gradients}
\begin{frame}{Model Gradients}
	\begin{itemize}
		\item Weight gradient:
	\end{itemize}
	\begin{align*}
	\dfrac{\partial E(x,w,b)}{\partial w} =& \dfrac{\partial}{\partial w} \frac{1}{n}\sum_{i = 0}^n \Big(y_i - (wx_i + b)\Big)^2\\
	& \frac{1}{n}\sum_{i = 0}^n \dfrac{\partial}{\partial w}\Big(y_i - (wx_i + b)\Big)^2\\
	& \frac{1}{n}\sum_{i = 0}^n 2x_i \Big(y_i - (wx_i + b)\Big)
	\end{align*}
\end{frame}

%%%%%%%%%%%%%%%%%%%%%%%%%%%%%%%%%%%%%%%%%%%%%%%%%%%%%%
%%%%%%%%%%%%%%%%%%%%%%%%%%%%%%%%%%%%%%%%%%%%%%%%%%%%%%
\begin{frame}{Model Gradients}
	\begin{itemize}
		\item Bias gradient:
	\end{itemize}
	\begin{align*}
	\dfrac{\partial E(x,w,b)}{\partial b} =& \dfrac{\partial}{\partial b} \frac{1}{n}\sum_{i = 0}^n \Big(y_i - (wx_i + b)\Big)^2\\
	& \frac{1}{n}\sum_{i = 0}^n \dfrac{\partial}{\partial b}\Big(y_i - (wx_i + b)\Big)^2\\
	& \frac{1}{n}\sum_{i = 0}^n 2 \Big(y_i - (wx_i + b)\Big)
	\end{align*}
\end{frame}

%%%%%%%%%%%%%%%%%%%%%%%%%%%%%%%%%%%%%%%%%%%%%%%%%%%%%%
%%%%%%%%%%%%%%%%%%%%%%%%%%%%%%%%%%%%%%%%%%%%%%%%%%%%%%
\subsection{Stochastic Gradient Descent}
\begin{frame}{Stochastic Gradient Descent}
	\begin{itemize}
		\item With that, our previous update equations become:
	\end{itemize}
	\begin{gather*}
	w \leftarrow w - \alpha \frac{1}{n}\sum_{i = 0}^n x_i \Big(y_i - (wx_i + b)\Big)\\
	b \leftarrow b - \alpha \frac{1}{n}\sum_{i = 0}^n \Big(y_i - (wx_i + b)\Big)
	\end{gather*}
	\begin{itemize}
		\item We don't need to go through all the $x$ in the data, instead we can randomly sample a much smaller \textbf{mini-batch} of $x$ at each step.
		\item Every time we go through the entire dataset, one mini-batch at a time, we finish an \textbf{epoch}.
	\end{itemize}
\end{frame}


%%%%%%%%%%%%%%%%%%%%%%%%%%%%%%%%%%%%%%%%%%%%%%%%%%%%%%
%%%%%%%%%%%%%%%%%%%%%%%%%%%%%%%%%%%%%%%%%%%%%%%%%%%%%%
\subsection{Intercept Data}
\begin{frame}{Intercept Data}
	\begin{itemize}
		\item We can use a small trick so that the gradients of $w$ and $b$ have the same form.
		\item We treat $x$ as a vector of features, such that in our polynomial model it represents $\boldsymbol{x} = [x^1, x^0]$.
		\item Mathematically $\boldsymbol{x}$ is just a vector, therefore the model becomes $y = \boldsymbol{w}\cdot\boldsymbol{x}$, where $\boldsymbol{w}$ represents our previous $[w, b]$.
		\item Our gradients now reduce to simply:
	\end{itemize}
	\begin{gather*}
	\dfrac{\partial E(x,w,b)}{\partial w} = \frac{1}{n}\sum_{i = 0}^n \boldsymbol{x}_i \cdot \Big(y_i - (\boldsymbol{w} \cdot \boldsymbol{x}_i)\Big)
	\end{gather*}
	\begin{itemize}
		\item \textbf{VECTORIZE CODE!}
	\end{itemize}
\end{frame}


%%%%%%%%%%%%%%%%%%%%%%%%%%%%%%%%%%%%%%%%%%%%%%%%%%%%%%
%%%%%%%%%%%%%%%%%%%%%%%%%%%%%%%%%%%%%%%%%%%%%%%%%%%%%%
\subsection{Pseudocode}
\begin{frame}{Pseudocode}
	\begin{algorithm}[H]
		\vspace{5px}
		\KwIn{Function $E(x, \boldsymbol{w})$ \newline
			  Learning rate $\alpha$}
		
		\vspace{10px}
		
		\Repeat{stopping condition}{
			Select random mini-batch $x'$\\
			$\boldsymbol{w} \leftarrow \boldsymbol{w} - \alpha \frac{\partial E(x',\boldsymbol{w})}{\partial \boldsymbol{w}}$ \\
		} 
	\end{algorithm}
\end{frame}

%%%%%%%%%%%%%%%%%%%%%%%%%%%%%%%%%%%%%%%%%%%%%%%%%%%%%%
%%%%%%%%%%%%%%%%%%%%%%%%%%%%%%%%%%%%%%%%%%%%%%%%%%%%%%
\section{\scshape Linear Regression}
\begin{frame}{Linear Regression}
	\begin{center}
		\animategraphics[controls,width=0.8\linewidth]{12}{lreg_anim/lreg_anim-}{0}{16}
	\end{center}
\end{frame}

%%%%%%%%%%%%%%%%%%%%%%%%%%%%%%%%%%%%%%%%%%%%%%%%%%%%%%
%%%%%%%%%%%%%%%%%%%%%%%%%%%%%%%%%%%%%%%%%%%%%%%%%%%%%%
\subsection{Minimum Found}
\begin{frame}{Optimization Solution}
	\begin{center}
		\includegraphics[width=0.9\columnwidth]{contour_plot}
	\end{center}
\end{frame}


%%%%%%%%%%%%%%%%%%%%%%%%%%%%%%%%%%%%%%%%%%%%%%%%%%%%%%
\section{\scshape Multiple Linear Regression}
\begin{frame}{Multiple Linear Regression}
	\begin{itemize}
		\item Using the same trick as the intercept data, we can also extend our model to higher order polynomials.
		\item Simply make it so that $\boldsymbol{x} = [x^n, ..., x^1, x^0]$
		\item A properly vectorized code should remain roughly the same.
		\item Generally, higher polynomials will have smaller costs.
	\end{itemize}
\end{frame}

%%%%%%%%%%%%%%%%%%%%%%%%%%%%%%%%%%%%%%%%%%%%%%%%%%%%%%
%%%%%%%%%%%%%%%%%%%%%%%%%%%%%%%%%%%%%%%%%%%%%%%%%%%%%%
\subsection{\scshape Fitted Polynomial}
\begin{frame}{Fifth Order Polynomial}
	\begin{center}
		\animategraphics[controls,width=0.8\linewidth]{12}{multiple_lreg_anim/mlreg-}{0}{12}
	\end{center}
\end{frame}

%%%%%%%%%%%%%%%%%%%%%%%%%%%%%%%%%%%%%%%%%%%%%%%%%%%%%%
%%%%%%%%%%%%%%%%%%%%%%%%%%%%%%%%%%%%%%%%%%%%%%%%%%%%%%
\subsection{\scshape Overfit}
\begin{frame}{Old Friend Overfit}
	\begin{itemize}
		\item Have to be careful not to use too high polynomials.
	\end{itemize}
	\begin{center}
		\includegraphics[width=0.85\columnwidth]{overfit}
	\end{center}
\end{frame}

%%%%%%%%%%%%%%%%%%%%%%%%%%%%%%%%%%%%%%%%%%%%%%%%%%%%%%
%%%%%%%%%%%%%%%%%%%%%%%%%%%%%%%%%%%%%%%%%%%%%%%%%%%%%%
\section{\scshape Questions}
\begin{frame}{Questions}
	\begin{enumerate}
		\item Why don't we invert matrices to perform linear regression?
		\item What is pointed by the gradient at any given point of a surface?
		\item How high should our learning rate ($\alpha$) be? Could you propose a way to reduce its sensitivity?
		\item Do we need to use our entire dataset at each gradient descent step?
		\item Suppose we want to predict a person's longevity and have the following data: Yearly health checkups, tobacco consumption, blood pressure levels, physical activity and university degree. Should we use all the data that is available to us?
		\item Can we perform linear regression from a strictly probabilistic perspective? 
	\end{enumerate}
\end{frame}

%%%%%%%%%%%%%%%%%%%%%%%%%%%%%%%%%%%%%%%%%%%%%%%%%%%%%%
%%%%%%%%%%%%%%%%%%%%%%%%%%%%%%%%%%%%%%%%%%%%%%%%%%%%%%
\begin{frame}{Maximum Likelihood Estimator}
	\begin{center}
		\includegraphics[width=\columnwidth]{MLE}\\[-1ex]
		{\tiny Credit: {\itshape Applied Linear Statistical Models 5e (2005)}}
	\end{center}
\end{frame}

%%%%%%%%%%%%%%%%%%%%%%%%%%%%%%%%%%%%%%%%%%%%%%%%%%%%%%
%%%%%%%%%%%%%%%%%%%%%%%%%%%%%%%%%%%%%%%%%%%%%%%%%%%%%%
\begin{frame}{Flavours of Gradient Descent}
	\begin{center}
		\animategraphics[controls,width=0.5\linewidth]{12}{gds_anim/gds-}{0}{118}\\
		{\tiny Credit: {\itshape \url{http://www.denizyuret.com/2015/03/alec-radfords-animations-for.html}}}
	\end{center}
\end{frame}
\end{document}